\documentclass[12pt]{article}

\usepackage{OSUDissertation}

\begin{document}

\begin{singlespace}

\begin{center}
AN ABSTRACT OF THE DISSERTATION OF
\end{center}

\vspace{12pt}

\noindent
\uline{\ThesisAuthor} for the degree of \uline{Doctor of Philosophy} in \uline{Nuclear Engineering} presented on \uline{\DefenseDate}.

\vspace{12pt}

\noindent
Title: \uline{Discrete Ordinates Radiation Transport using High-Order Finite Element Spatial Discretizations on Meshes with Curved Surfaces}

\vspace{12pt}

\noindent
Abstract approved:

\vspace{12pt}

\noindent
\begin{center}
\rule{\textwidth}{0.4pt}

\noindent
Todd S. Palmer
\end{center}

\vspace{12pt}

\end{singlespace}

We spatially discretize the \SN\ transport equation using the high-order (HO) discontinuous finite element method (DFEM) on HO meshes. Previous work provided a proof-of-concept for this spatial discretization method in \XY\ geometry. Included in the present work, we derive a spatial discretization for the \SN\ transport equation in both \XY\ and \RZ\ geometries using HO DFEMs on HO meshes. We characterize the behavior of these methods by solving several numerical test problems. In \XY\ geometry, we determine that the discretization error dominates the errors introduced by the approximation of the spatial integrals on the surface term on problems with curved surfaces that are both incident and outgoing. We also conclude for an optically thick and diffusive problem, that these \emph{highly} curved meshes may propagate the boundary layer oscillations into the problem interior. In \RZ\ geometry, we conclude that the spatial discretization error is dependent upon the angular discretization, that the spatial error converges at $O(p+1)$ on problems with smooth solutions, that the spatial accuracy degrades for solutions with less regularity, and that preserving a particular spherical symmetry problem is primarily a function of finite element order and spatial mesh resolution.

This work also implements the modified interior penalty (MIP) diffusion synthetic acceleration (DSA) method and investigates the sensitivities of the spectral radius to several parameters, including finite element order, mesh curvature, and a user-defined constant. We conclude that the spectral radius is especially dependent upon the finite element order and the user-defined constant. We also derive and implement a Robin boundary condition for the MIP DSA method to accelerate the source iteration. The testing we perform reveals substantially faster convergence rates in the optically thin and intermediate regimes, while degrading the convergence rates in optically thick regimes when compared to the original MIP DSA equations. Also, we conclude that the Robin boundary condition method reduces the dependence of the spectral radius on changes in both the finite element order and the user-defined constant.

\end{document}