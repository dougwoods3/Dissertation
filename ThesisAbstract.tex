\documentclass[12pt]{article}

\usepackage{OSUDissertation}

\begin{document}

\begin{singlespace}

\begin{center}
AN ABSTRACT OF THE DISSERTATION OF
\end{center}

\vspace{12pt}

\noindent
\uline{\ThesisAuthor} for the degree of \uline{Doctor of Philosophy} in \uline{Nuclear Engineering} presented on \uline{\DefenseDate}

\vspace{12pt}

\noindent
Title: \uline{\ThesisTitle}

\vspace{12pt}

\noindent
Abstract approved:

\vspace{12pt}

\noindent
\begin{center}
\rule{\textwidth}{0.4pt}

\noindent
Todd S. Palmer
\end{center}

\vspace{12pt}

\end{singlespace}

{\color{red}
The high-order finite element \SN transport equations are solved on several test problems to investigate the behavior of the discretization method on meshes with curved edges in X-Y geometry. Simpler problems ensured the correct implementation of MFEM, the general finite element library employed. A convergence study using the method of manufactured solutions demonstrates the convergence rate as a function of number of unknowns for meshes whose sides are described by polynomial curves of increasing order. Optically thick and diffusive problems indicate the DGFEM transport solution trends toward a numerical solution of the diffusion equation, though a rigorous diffusion limit analysis has not yet been performed. A direct solve approach to computing the angular flux unknowns simultaneously is presented as an alternative to source iteration that involves linear algebraically inverting the ``streaming plus collision minus scattering'' operator. These results serve as a proof-of-concept of this spatial discretization method.
}

\end{document}