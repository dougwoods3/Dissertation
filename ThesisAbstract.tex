\documentclass[12pt]{article}

\usepackage{OSUDissertation}

\begin{document}

\begin{singlespace}

\begin{center}
AN ABSTRACT OF THE DISSERTATION OF
\end{center}

\vspace{12pt}

\noindent
\uline{\ThesisAuthor} for the degree of \uline{Doctor of Philosophy} in \uline{Nuclear Engineering} presented on \uline{\DefenseDate}

\vspace{12pt}

\noindent
Title: \uline{\ThesisTitle}

\vspace{12pt}

\noindent
Abstract approved:

\vspace{12pt}

\noindent
\begin{center}
\rule{\textwidth}{0.4pt}

\noindent
Todd S. Palmer
\end{center}

\vspace{12pt}

\end{singlespace}

We spatially discretize the \SN\ transport equation using the high-order (HO) discontinuous finite element method (DFEM) on HO meshes. Our previous work provided a proof-of-concept for this spatial discretization method in \XY\ geometry. Included in the present work, we derive a spatial discretization for the \SN\ transport equation in \RZ\ geometry using HO DFEM on HO meshes. We characterize the behavior of these methods by solving several numerical test problems. We conclude the spatial discretization error is dependent upon the angular discretization, the spatial error converges at $O(p+1)$ on problems with smooth solutions, the spatial error converges at $O(3/2)$ for solutions with less regularity, and preserving axisymmetry is a function of finite element order and spatial mesh resolution. Specifically, our method preserves the one-dimensional spherical geometry solution using a smooth manufactured solution.

We also derive and implement a vacuum boundary condition for the modified interior penalty (MIP) diffusion synthetic acceleration (DSA) method to accelerate the source iteration. The testing we perform reveals substantially faster convergence rates in the optically thin and intermediate regimes, while degrading the convergence rates in optically thick regimes when compared to the original MIP DSA equations.

\end{document}