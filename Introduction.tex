\documentclass[12pt,letterpaper]{article}
\usepackage{OSUDissertation}

\begin{document}
%%%%%%%%%%%%%%%%%%%%%%%%%%%%%%%%%%%%%%%
\section{Introduction}
The solution to the radiation transport equation is valuable in the design and analysis of many physical systems. A few examples are nuclear reactors~\cite{D&H}, medical radiation treatments~\cite{Sandwall2018Brachytherapy}, and high energy density physics systems like those found in astrophysics or inertial confinement fusion (ICF)~\cite{Castor_Rad_Hydro}. ICF experiments are being performed at facilities such as the National Ignition Facility\footnote{\url{https://lasers.llnl.gov}} (NIF) and the Omega Laser Facility\footnote{\url{http://www.lle.rochester.edu/omega_facility/omega/}}. Nuclear fusion occurs in the high energy-density physics (HEDP) regime where the energy-density of a material is very high~\cite{Castor_Rad_Hydro,DrakeHEDPPaper}. If material temperatures are very high, materials emit black body radiation (also called thermal radiation) in tremendous quantities and absorb this thermal radiation depositing energy back to the material influencing the material internal energy, momentum, temperature, and density. This complicated system of thermal radiation transport with hydrodynamics is the subject of the field of radiation hydrodynamics. While all of the forces in these HEDP systems are important, we can develop some of the computational models independently. For instance, we may develop hydrodynamics models (and software tools) separately from those for radiation transport.

Lawrence Livermore National Laboratory (LLNL) is developing a hydrodynamics code called BLAST\footnote{\url{https://computation.llnl.gov/project/blast/}}~\cite{DobrevHOFEMHydro}. BLAST solves the Euler equations using a general finite element method (FEM) on meshes with curved surfaces for the conservation of mass, energy, and momentum of a fluid~\cite{DobrevCurvilinearFEMHydro}. Novel features include higher-order elements to represent the thermodynamic and kinematic variables, and meshes with curved surfaces. Compared to the most common method, staggered grid hydrodynamics, Dobrev et al.~\cite{DobrevCurvilinearFEMHydro} demonstrated that BLAST can more accurately model flow geometry, symmetry of radial flow, and precise location of a shock front --- observing that the FEM approach has the ability to model the shock within a single mesh element. These improvements have reduced some of the numerical errors that have been exhibited by previous methods.

BLAST employs the Modular Finite Element Methods Library\footnote{\url{mfem.org}} (MFEM)~\cite{MFEM_Web}, a general finite element library also being developed at LLNL, to spatially discretize using the high-order (HO) FEM on meshes with curved surfaces. To  become an effective radiation hydrodynamics package, BLAST must be integrated with a radiation transport package that is able to model the radiation field in these HEDP regimes. One approach to this coupling is to use the same mesh and spatial discretization as the hydrodynamics being modeled. The main topic of this thesis is rooted in the spatial discretization of the radiation transport equation using HO finite elements on HO meshes.

This chapter introduces the radiation transport equation and establishes the motivation for using the HO methods. In Section~\ref{sec:RadiationTransportIntro}, we give a brief introduction to the radiation transport equation and nomenclature. In Section~\ref{sec:DiscretizationIntro}, we describe various discretization methods used to solve the radiation transport equation.
%In Section~\ref{sec:RZGeometryIntro}, we introduce the research that has been performed solving the transport equation using \RZ\ geometry.
In Section~\ref{sec:SphericalSymmetryIntro}, we further the \RZ\ geometry discussion by motivating a need for solving a spherically symmetric system using \RZ\ geometry. In Section~\ref{sec:DiffLimitIntro}, we describe the diffusion limit and its application to optically thick problems found in HEDP regimes.  In Section~\ref{sec:SourceIterationAccelerationIntro}, we discuss some of the research conducted for accelerating the source iteration method. In Section~\ref{sec:ResearchObjectivesIntro}, we state the research objectives for this dissertation. Finally, in Section~\ref{sec:OutlineIntro}, we outline the remainder of this dissertation.

%%%%%%%%%%%%%%%%%%%%%%%%%%%%%%%%%%%%%%%
\subsection{Radiation Transport}
\label{sec:RadiationTransportIntro}
The time- and energy-dependent radiation transport equation,
\begin{multline}
\frac{1}{u} \frac{\partial \psi (\vec{x},\vec{\Omega},E,t)}{\partial t} + \vec{\Omega} \vd \grad \psi(\vec{x},\vec{\Omega},E,t) + \sigma_t(\vec{x},\vec{\Omega},E,t) \psi(\vec{x},\vec{\Omega},E,t) \\
= \frac{1}{4 \pi} \int_{4 \pi} \int_0^\infty \sigma_s(\vec{x},\vec{\Omega}^\prime \rightarrow \vec{\Omega},E^\prime \rightarrow E,t) \psi(\vec{x},\vec{\Omega}^\prime ,E^\prime,t)\ dE^\prime\ d \Omega^\prime + S_0(\vec{x},\vec{\Omega},E,t),
\label{eq:FullDependentTransport}
\end{multline}
%
\noindent describes the angular flux distribution, $\psi(\vec{x},\vec{\Omega},E,t)$, as a function of position, $\vec{x}$, direction of travel, $\vec{\Omega}$, energy, $E$, and time, $t$. We also define the scalar flux by
\begin{flalign}
\phi(\vec{x},E,t) & = \int_{4 \pi} \psi(\vec{x},\vec{\Omega},E,t)\ d \Omega,
\label{eq:ScalarFluxIntegral}
\end{flalign}
%
\noindent that describes the density of particles traveling in all directions at each spatial position $\vec{x}$. Both the angular and scalar fluxes describe a particle density in a phase space denoted by their dependent variables. The angular flux has units $\text{cm}^{-2}\ \text{s}^{-1} \text{ ster}^{-1}$. The scalar flux is integrated over all directions and has units $\text{cm}^{-2}\ \text{s}^{-1}$. In general, the total macroscopic cross section, $\sigma_t(\vec{x},\vec{\Omega},E,t)=\varsigma_t(\vec{x},\vec{\Omega},E,t) N(\vec{x},t)$, is a property of the microscopic total cross section, $\varsigma_t(\vec{x},\vec{\Omega},E,t) \text{ cm}^{2}$, and the number density of the material, $N(\vec{x},t) \text{ cm}^{-3}$. The absorption and scattering cross sections are related to the total cross section by $\sigma_t(\vec{x},\vec{\Omega},E,t)=\sigma_a(\vec{x},\vec{\Omega},E,t) + \sigma_s(\vec{x},\vec{\Omega},E,t)$. The value $\sigma_s(\vec{x},\vec{\Omega}^\prime \rightarrow \vec{\Omega},E^\prime \rightarrow E,t)$ denotes scattering from direction $\vec{\Omega}^\prime$ into $\vec{\Omega}$ and from energy $E^\prime$ into $E$ at position $\vec{x}$ and time $t$. Additionally, $S_0(\vec{x}, \vec{\Omega},E,t)$ is some radiation emitting external source (e.g., fission neutrons or blackbody radiation).

There are only a few specific cases where the radiation transport equation can be solved exactly. Problems of interest typically fall outside of this subset and numerical methods must be employed to obtain approximate solutions. There are two general categories of methods used to solve this equation: stochastic and deterministic methods.

Stochastic methods take a statistical approach. In any particular source region, a single particle is emitted and tracked through its ``random walk'', traversing the problem domain until it is either absorbed or escapes the domain. One can then simply tally the number of particles that get absorbed into a region of interest. This method is rooted in the physics of the stochastic particle-material interactions. If enough particles are simulated, a statistically significant conclusion can be drawn about the particle density in a region of interest. For simple problems (absent of nonlinear effects like particle-particle collisions), this can be very accurate but also computationally demanding because of the large number of particles required to be simulated.

Alternatively, deterministic methods solve the radiation transport equation for the angular and scalar fluxes by discretizing the equation in each of the dependent variables (i.e., space and direction of travel). The discretized equation results in a linear system of equations that can be solved with linear algebra techniques. Deterministic methods are generally faster than stochastic methods but involve discretization or truncation errors. However, employing certain discretizations can reduce the impact of these errors.

%%%%%%%%%%%%%%%%%%%%%%%%%%%%%%%%%%%%%%%
\subsection{Discretization}
\label{sec:DiscretizationIntro}
Physically, particles can be distributed continuously in three-dimensional space. Numerically, we discretize the spatial domain into a smaller subset of discrete locations. Similarly, a particle's direction of travel can be distributed continuously so we approximate it traveling in only a few discrete directions. It is common to discretize the direction of travel into a finite number of directions by using the discrete ordinates (\SN) method~\cite{discrete_ordinates}. Particles with various energies may behave or interact with materials differently. It is common to approximate the continuum of particle energies by dividing it into a finite number of energy groups. If a problem evolves through time, we break the time continuum into discrete time steps. Ideally, we would model the continuous phase space because it describes the physical behavior of neutral particles. However, we discretize each of the continuous variables to approximate each as closely as practical. In general, this can mean refining the discretization, e.g., refining the spatial mesh or using a higher order discrete ordinates approximation. However, the cost of higher accuracy is increased computational time that may become prohibitively expensive.

The transport equation discretized in time, energy, and direction can be written
\begin{multline}
\frac{1}{u} \frac{\partial \psi (\vec{x},\vec{\Omega}_m,E_g,t_\tau)}{\partial t} + \vec{\Omega} \vd \grad \psi(\vec{x},\vec{\Omega}_m,E_g,t_\tau) \\
+ \sigma_t(\vec{x},\vec{\Omega}_m,E_g,t_\tau) \psi(\vec{x},\vec{\Omega}_m,E_g,t_\tau) \\
= \frac{1}{4 \pi} \int_{4 \pi} \int_0^\infty \sigma_s(\vec{x},\vec{\Omega}^\prime \rightarrow \vec{\Omega}_m,E^\prime \rightarrow E_g,t_\tau) \psi(\vec{x},\vec{\Omega}^\prime ,E^\prime,t_\tau)\ dE^\prime\ d \Omega^\prime \\
+ S_0(\vec{x},\vec{\Omega}_m,E_g,t_\tau),
\label{eq:FullDependentTransport}
\end{multline}
%
\noindent and
\begin{flalign}
\phi(\vec{x},E_g,t_\tau) & = \int_{4 \pi} \psi(\vec{x},\vec{\Omega}^\prime,E_g,t_\tau)\ d \Omega^\prime,
\label{eq:ScalarFluxIntegral}
\end{flalign}
%
A flow chart of the general solution algorithm is shown in Figure~\ref{fig:SolutionFlowDiagram}.
%
\begin{figure}
\small
\begin{tikzpicture}[node distance=2cm]
\tikzstyle{startstop} = [rectangle, rounded corners, minimum width=3cm, minimum height=1cm, text width=4cm, text centered, draw=black, fill=red!30];
\tikzstyle{io} = [trapezium, trapezium left angle=70, trapezium right angle=110, minimum width=1cm, minimum height=1cm, text width=1.5cm, text centered, draw=black, fill=blue!30];
\tikzstyle{process} = [rectangle, minimum width=3cm, minimum height=1cm, text centered, text width=3cm, draw=black, fill=orange!30];
\tikzstyle{decision} = [rectangle, minimum width=3cm, minimum height=1cm, text width=3cm, text centered, draw=black, fill=green!30];
\tikzstyle{arrow} = [thick,->,>=stealth];

\node (timeSteps) [startstop] {Advance through all time steps\\$t_\tau=t_0, t_1, t_2, \dots$};
\node (energyGroups) [startstop, below right of=timeSteps, xshift=1.5cm,  yshift=-0.75cm] {Loop over all energy groups\\$g=1,\dots,G$};
\node (angleDiscretize) [startstop, below right of=energyGroups, xshift=1.5cm, yshift=-1cm] {Angular discretization};
\node (angleDirs) [startstop, below of=angleDiscretize] {Loop over all discrete ordinate directions\\$m=1, \dots, M$};
\node (spaceDiscretize) [startstop, below right of=angleDirs, xshift=1.5cm, yshift=-0.5cm] {Spatial discretization};
\node (linTransport) [process, below of=spaceDiscretize, yshift=0.25cm] {Radiation transport solve};
\node (EndAngleDirs) [decision, below left of=linTransport, xshift=-1.5cm] {End of $m$ loop?};
\node (DSAUpdate) [process, below of=EndAngleDirs] {DSA solve [optional]};
\node (updateScalarFlux) [process, below of=DSAUpdate, yshift=0.25cm] {Update scalar flux};
\node (ScalarFluxConv) [decision, below of=updateScalarFlux, yshift=0.25cm] {Scalar flux converged?};
\node (EndEnergyGroups) [decision, below left of=ScalarFluxConv, xshift=-1.5cm, yshift=-1cm] {End of $g$ loop?};

\draw [arrow] (timeSteps) -| (energyGroups);
\draw [arrow] (energyGroups) -| (angleDiscretize);
\draw [arrow] (angleDiscretize) -- (angleDirs);
\draw [arrow] (angleDirs) -| (spaceDiscretize);
\draw [arrow] (spaceDiscretize) -- (linTransport);
\draw [arrow] (linTransport) |- (EndAngleDirs);
\draw [arrow] (EndAngleDirs) node[anchor=east, yshift=-1cm]{yes} -- (DSAUpdate);
\draw [arrow] (DSAUpdate) -- (updateScalarFlux);
\draw [arrow] (updateScalarFlux) -- (ScalarFluxConv);
\draw [arrow] (EndAngleDirs) node[anchor=east, yshift=1cm]{no} |- ($(spaceDiscretize.north) + (0,0.35cm)$);
\draw [arrow] (ScalarFluxConv) node[anchor=north, xshift=-2.25cm]{no} -- ++ (-2.5cm,0) -- ($(angleDirs) + (-2.5cm,0)$) -- (angleDirs);
\draw [arrow] (ScalarFluxConv) node[anchor=west, yshift=-1.5cm]{yes} |- (EndEnergyGroups);
\draw [arrow] ($(EndEnergyGroups.north) + (-0.25cm,0)$) node[anchor=east, yshift=0.5cm]{no} |- ($(angleDiscretize.north) + (0,0.75cm)$);
\draw [arrow] (EndEnergyGroups) node[anchor=north east, xshift=-2cm]{yes} -| (timeSteps);

\draw [red,ultra thick, dashed] (3,-3.6) rectangle (11.2,-18);

\end{tikzpicture}
\caption{Flow diagram for solution process. This research is focused on the processes boxed by the dashed line.}
\label{fig:SolutionFlowDiagram}
\end{figure}
%
Discretizing the transport equation in all of phase space, we first advance through each time step ($t_\tau=t_0, t_1, \dots$). Within each time-step, we loop through each energy group ($g=1, \dots, G$). Within each energy group, we loop through all discrete ordinate directions ($m=1, \dots, M$). Within the angular discretization loop, we spatially discretize the transport equation, solve it, and obtain a scalar flux. We perform a source iteration until the scalar flux is converged.

This research is concerned with the innermost loop (boxed in Figure~\ref{fig:SolutionFlowDiagram} with a dashed line) over discrete ordinate directions until the scalar flux converges. We assume the solution of the transport equation has reached steady-state (or is being solved at a single time-step), making the problem time independent. We also assume there is only one energy group (or we are solving for one of many energy groups), thereby making the problem energy independent. We discretize the two remaining dependent variables (space and direction of travel) where the spatial discretization is the main focus of this dissertation. The steady-state, mono-energetic radiation transport equation,
\begin{flalign}
\vec{\Omega} \vd \grad \psi(\vec{x},\vec{\Omega}) + \sigma_t(\vec{x}) \psi(\vec{x},\vec{\Omega}) & = \frac{1}{4 \pi} \sigma_s(\vec{x}) \int_{4 \pi} \psi(\vec{x},\vec{\Omega}^\prime) d \Omega^\prime + \frac{1}{4 \pi} S_0(\vec{x}),
\label{eq:RadTransport}
\end{flalign}
%
\noindent describes the angular flux distribution, $\psi(\vec{x},\vec{\Omega})$, as a function of position, $\vec{x}$, and direction of travel, $\vec{\Omega}$. For the present work, we assume the cross sections are only dependent upon the spatial position. The scalar flux,
\begin{flalign}
\phi(\vec{x}) & = \int_{4 \pi} \psi(\vec{x}, \vec{\Omega}^\prime) d \Omega^\prime,
\label{eq:ScalarFluxIntegral}
\end{flalign}
%
\noindent describes the density of particles traveling in all directions at every point $\vec{x}$. Hereafter, we have dropped the energy and time variables from the arguments for brevity.

%%%%%%%%%%%%%%%%%%%%%%%%%%%%%%%%%%%%%%%
\subsubsection{Direction of Travel Discretization}
\label{subsec:DirectionOfTravelDiscretization}
We approximate the continuum of directions of particle travel by assuming they only travel in discrete ordinate ($S_N$) directions. Figure~\ref{fig:DiscreteOrdinates} shows an example of a set of discrete ordinates.
%
\begin{figure}[!htb]
\centering
\tdplotsetmaincoords{60}{110}
\begin{tikzpicture}[scale=8,tdplot_main_coords]
\pgfmathsetmacro{\rvec}{.8}
\pgfmathsetmacro{\phivec}{40}
\pgfmathsetmacro{\thetavec}{50}
\pgfmathsetmacro{\omegavec}{60}

\coordinate (O) at (0,0,0);
\draw[thick,->] (0,0,0) -- (1.1,0,0) node[anchor=north east]{$\mu$};
\draw[thick,->] (0,0,0) -- (0,1.1,0) node[anchor=north west]{$\eta$};
\draw[thick,->] (0,0,0) -- (0,0,1.1) node[anchor=south]{$\xi$};

\tdplotdrawarc[dashed]{(0,0,0)}{1}{0}{90}{anchor=north west}{}
%
\tdplotsetthetaplanecoords{0}
\tdplotsetrotatedcoords{270}{270}{0}
\tdplotsetrotatedcoordsorigin{(O)}
\tdplotdrawarc[tdplot_rotated_coords,dashed]{(0,0,0)}{1}{0}{90}{anchor=south west}{}
%
\tdplotsetrotatedcoords{0}{270}{0}
\tdplotsetrotatedcoordsorigin{(O)}
\tdplotdrawarc[tdplot_rotated_coords,dashed]{(0,0,0)}{1}{0}{90}{anchor=west}{}

% constant \xi
\tdplotsetrotatedcoords{0}{0}{0}
\tdplotsetrotatedcoordsorigin{(O)}
\tdplotdrawarc[tdplot_rotated_coords,dashed]{(0,0,0.266)}{0.964}{0}{90}{anchor=east}{}
\tdplotsetrotatedcoords{0}{0}{0}
\tdplotsetrotatedcoordsorigin{(O)}
\tdplotdrawarc[tdplot_rotated_coords,dashed]{(0,0,0.6815)}{0.7318}{0}{90}{anchor=west}{}
\tdplotsetrotatedcoords{0}{0}{0}
\tdplotsetrotatedcoordsorigin{(O)}
\tdplotdrawarc[tdplot_rotated_coords,dashed]{(0,0,0.926)}{0.3775}{0}{90}{anchor=west}{}

% constant \mu
\tdplotsetrotatedcoords{0}{90}{90}
\tdplotsetrotatedcoordsorigin{(O)}
\tdplotdrawarc[tdplot_rotated_coords,dashed]{(0,0,0.266)}{0.964}{0}{90}{anchor=east}{}
\tdplotsetrotatedcoords{0}{90}{90}
\tdplotsetrotatedcoordsorigin{(O)}
\tdplotdrawarc[tdplot_rotated_coords,dashed]{(0,0,0.6815)}{0.7318}{0}{90}{anchor=west}{}
\tdplotsetrotatedcoords{0}{90}{90}
\tdplotsetrotatedcoordsorigin{(O)}
\tdplotdrawarc[tdplot_rotated_coords,dashed]{(0,0,0.926)}{0.3775}{0}{90}{anchor=west}{}

% constant \eta
\tdplotsetrotatedcoords{270}{270}{0}
\tdplotsetrotatedcoordsorigin{(O)}
\tdplotdrawarc[tdplot_rotated_coords,dashed]{(0,0,0.266)}{0.964}{0}{90}{anchor=east}{}
\tdplotsetrotatedcoords{270}{270}{0}
\tdplotsetrotatedcoordsorigin{(O)}
\tdplotdrawarc[tdplot_rotated_coords,dashed]{(0,0,0.6815)}{0.7318}{0}{90}{anchor=west}{}
\tdplotsetrotatedcoords{270}{270}{0}
\tdplotsetrotatedcoordsorigin{(O)}
\tdplotdrawarc[tdplot_rotated_coords,dashed]{(0,0,0.926)}{0.3775}{0}{90}{anchor=west}{}

% discrete ordinate nodes
\node[circle,draw,fill] at (0.2666,0.2666,0.92618){};
\node[circle,draw,fill] at (0.2666,0.6815,0.6815){};
\node[circle,draw,fill] at (0.2666,0.9262,0.2666){};
\node[circle,draw,fill] at (0.6815,0.2666,0.6815){};
\node[circle,draw,fill] at (0.6815,0.6815,0.2666){};
\node[circle,draw,fill] at (0.9262,0.2666,0.2666){};

\end{tikzpicture}
\caption{The positive octant for $S_6$ level-symmetric angular quadrature.}
\label{fig:DiscreteOrdinates}
\end{figure}
%
Specifically, this octant of the $S_6$ level-symmetric quadrature demonstrates that $\vec{\Omega}$ is discretized into discrete directions $\vec{\Omega}_m$ and can be described as a function of $\mu$, $\eta$, and $\xi$, the three coordinate axes shown in Figure~\ref{fig:DiscreteOrdinates}.

We utilize the discrete ordinate weights to approximate the integral in Equation~\ref{eq:ScalarFluxIntegral} by
\begin{flalign}
\phi & \approx \sum_m^M w_m \psi_m,
\label{eq:SnWeightedSum}
\end{flalign}

\noindent where $M=N(N+2)$. This approximation is utilized in these numerical methods outlined below. The set of angular quadrature data includes the $\mu_m$, $\eta_m$, $\xi_m$, and $w_m$ values for each discrete ordinate direction $m$ for $m=1, \dots, M$, and we define $\psi_m \equiv \psi (\vec{x},\vec{\Omega}_m)$. It is possible to generate these data sets~\cite{Spence2015Quadrature} but we employ pre-generated data sets from ARDRA,\footnote{\url{https://wci.llnl.gov/simulation/computer-codes}} a \SN\ radiation transport code being developed at LLNL.

%%%%%%%%%%%%%%%%%%%%%%%%%%%%%%%%%%%%%%%
\subsubsection{Spatial Discretization}
\label{subsec:SpatialDiscretization}
There are countless methods of spatial discretization that appear in the transport community. Among the most common are characteristic methods~\cite{AdamsCharacteristicMethods}, finite difference methods~\cite{Lewis_Comp_Methods_Neu_Trans}, finite volume methods~\cite{Palmer2016Neutronics4}, and finite element methods (FEM)~\cite{Lewis_Comp_Methods_Neu_Trans}. Typically, the problem domain is divided into a larger number of smaller domains. The equations are then numerically solved on each of these smaller regions where the solution is likely to be less varying when compared to the entire problem. The FEM was introduced to the transport community in the 1970's. The utility of this method was that it was more accurate than finite differencing methods~\cite{ReedTriangularMesh}. Since then, radiation transport research using the FEM has proliferated. In particular, the discontinuous FEM (DFEM) has been favored for its accuracy in the thick diffusion limit~\cite{LarsenAsymptotic} (discussed in Section~\ref{sec:DiffLimitIntro}). Thus, this research uses the DFEM where the solution is approximated to have a functional form within each mesh element. More discussion of various discretization methods can be found in Lewis and Miller~\cite{Lewis_Comp_Methods_Neu_Trans}.

Additionally, these radiation transport problems are solved in Cartesian, cylindrical, and spherical coordinates in one-, two-, and three-dimensions. The present research is concerned with \XY\ and \RZ\ geometries ---  two-dimensional Cartesian and cylindrical geometries, respectively.

%%%%%%%%%%%%%%%%%%%%%%%%%%%%%%%%%%%%%%%
\subsubsection{High-Order Finite Elements}
\label{sec:HODFEMIntro}
The FEM spatial discretization for the radiation transport equation has been continually investigated since the 1970s~\cite{ReedTriangularMesh, LasaintFEM}. Although FEMs require more computer memory than finite-difference methods for the same spatial mesh, low-order (LO) methods, such as the linear discontinuous FEM (LD), have been used because of their increased accuracy~\cite{LarsenAsymptotic} and continue to be investigated~\cite{LarsenConvergenceRates,HamiltonNegativeFluxFixups,Adams_Disc_FEM_Thick_Diff}. Studying and implementing more accurate methods becomes increasingly important as computational hardware performance increases. Computers can process and communicate more quickly, and larger memory can store more variables. LO FEMs have been used and shown to be accurate in production codes like Attila~\cite{WareingAttila}, a three-dimensional transport code. Attila later adopted Trilinear DFEM (TLD), the three-dimensional analog to bilinear DFEM (BLD)~\cite{AttilaUsersManual}. Other methods have been studied as well, such as the piece-wise linear DFEM (PWLD)~\cite{BaileyDFEMCylindrical}.

Higher order DFEMs (polynomial orders $p \geq 2$) have become more popular as modern computer performance improves. Several authors~\cite{WangRagusaDSA, WangHODGTransport, WangDGFEMConvergence, WangDissertation} saw increased accuracy for up to 4$^\text{th}$-order finite elements over lower-order methods with the same mesh refinement. Woods et al.~\cite{WoodsHoDgfemXyCurved} saw increased accuracy for up to 8$^\text{th}$-order finite elements. These results further motivate the need for additional investigation of these HO methods.

Negative solutions can arise from oscillations within the HO FEM when the solution is near zero~\cite{WoodsThesis,WoodsHoDgfemXyCurved}. These negative solutions are non-physical and may require amelioration. Several authors have addressed negative scalar fluxes~\cite{HamiltonNegativeFluxFixups, Adams_Disc_FEM_Thick_Diff, MaginotNonNegative, MaginotLumpingDFEM, BrunnerPreservingPositivity} with corrective methods generally falling into two categories: \emph{ad hoc} modification after calculating a solution, and modification of the equations beforehand to yield positive results. In this dissertation, we characterize the behavior of the HO methods on meshes with curved surfaces and retain the negative fluxes as results of this methodology. With favorable results supporting continued analysis, we will leave the relevant investigation of negative flux corrections to future work.

%Maginot et al.~\cite{MagniotHighOrderLumping} determined traditional mass-matrix lumping for one-dimensional high order methods do not increase in accuracy with an increase in finite element order. 

% We are likely to be concerned with negative fluxes when integrating the linear transport equation with the material temperature equation. When appropriate, we will acknowledge the negative fluxes but retain them as part of the methodology~\cite{GreshoDontSuppressWiggles}.

%%%%%%%%%%%%%%%%%%%%%%%%%%%%%%%%%%%%%%%
\subsubsection{Meshes}
There are a wide variety of meshes in use with the transport community, with triangular and quadrilateral (and their three-dimensional equivalents) being the most common. Modeling curved boundaries with quadrilateral spatial meshes can be difficult without significant mesh refinement. Triangular meshes may be used in these cases and are often seen in literature~\cite{ReedTriangularMesh,WangHODGTransport,WangDGFEMConvergence, MorelLLDrz}. Additionally, unstructured meshes are popular for modeling complex spatial geometries. For example, in three-dimensions, Attila can discretize the spatial domain using unstructured tetrahedral meshes~\cite{WareingAttila}.
%The PWLD mesh refinement truncation error was first thought to behave as well on unstructured meshes as other methods using structured grids~\cite{StonePLFEM}.

Recently, meshes with curved surfaces have been investigated using finite elements for radiation transport. Liu and Larsen~\cite{Liu2015SNCurvedMesh} used meshes with circular arcs to model a fuel pin geometry exactly. They performed their calculations using an average outward normal direction for the zone surface and swept through the mesh with surface average fluxes equal to the zone average fluxes. Schunert et al.~\cite{Schunert2017HOMeshes} used $1^\text{st}$-order FEM but with $2^\text{nd}$-order polynomial mesh surfaces instead of circular arcs for their fuel lattice geometry. They demonstrated an increase in accuracy using the higher-order mesh without an increase in computational time. Woods~\cite{WoodsThesis} and Woods et al.~\cite{WoodsHoDgfemXyCurved} demonstrated the use of up to $8^\text{th}$-order meshes. The motivation behind using meshes with curved surfaces is to model curved boundaries within a problem without requiring additional mesh refinement. Woods et al.~\cite{WoodsHoDgfemXyCurved} performed a spatial convergence study using meshes with curved surfaces and saw convergence rates of $(p+1)$ consistent with predictions and results from Lasaint and Raviart~\cite{LasaintFEM} and Wang and Ragusa~\cite{WangHODGTransport}.

% Get some fluid stuff in here (from Veselin) about them using curved surfaces so it warrants research in the radiation transport regime. THIS IS WHAT MAKES THIS RESEARCH RELEVANT, NEW, AND UNIQUE.

Other fields have investigated the use of meshes with curved surfaces. Cheng~\cite{ChengCurvMeshEulerEqs} demonstrated that straight-edged meshes, compared to curved meshes, restricted the accuracy of the solution of the compressible Euler equations, pointing to the necessity of curved meshes for higher-order accuracy. Dobrev et al.~\cite{DobrevHOFEMHydro} saw increased accuracy using such meshes for Lagrangian hydrodynamics. They mapped the reference element to the physical element using the basis functions that resulted in curved surfaces for HO finite elements. Subsequently, Dobrev et al.~\cite{DobrevHOAxisymmetric} demonstrated the use of curved meshes in axisymmetric geometries and noted that they modeled some features of the flow more accurately, among other improvements.

%%%%%%%%%%%%%%%%%%%%%%%%%%%%%%%%%%%%%%%
\subsubsection{\RZ\ Geometry}
\label{sec:RZGeometryIntro}
The radiation transport equation has been solved in \RZ\ (i.e., two-dimensional cylindrical) geometry for many years. One particular difficulty with cylindrical coordinates is that an angular derivative is present within the streaming term. This results in the description of the direction-of-travel being dependent upon the spatial position. Despite a particle traveling in a straight line, the coordinate system describing that direction changes with position. 

The streaming operator in three-dimensional cylindrical geometry is~\cite{Lewis_Comp_Methods_Neu_Trans}
\begin{flalign}
\vec{\Omega} \vd \grad \psi & = \frac{\mu}{r} \frac{\partial}{\partial r} (r \psi) + \frac{\eta}{r} \frac{\partial \psi}{\partial \zeta} + \xi \frac{\partial \psi}{\partial z} - \frac{1}{r} \frac{\partial}{\partial \omega} (\eta \psi),
\end{flalign}
%
where $\vec{\Omega}$ is the direction-of-travel unit vector, $\psi$ is the angular flux, and
\begin{flalign}
\mu & \equiv \vec{\Omega} \vd \hat{e}_\mu = \sqrt{1 - \xi^2} \cos \omega = \sin(\theta) \cos(\omega), \label{eq:muDef} \\
\eta & \equiv \vec{\Omega} \vd \hat{e}_\eta = \sqrt{1 - \xi^2} \sin \omega = \sin(\theta) \sin(\omega), \label{eq:etaDef} \\
\xi & \equiv \vec{\Omega} \vd \hat{e}_\xi = \cos(\theta). \label{eq:xiDef}
\end{flalign}
%
The variables $\mu$, $\eta$, $\xi$, $\omega$, $\theta$, and $\zeta$ are shown in the cylindrical coordinate system in Figure~\ref{fig:CylindricalCoordinateSystem}.
%
\begin{figure}[!h]
\tdplotsetmaincoords{60}{110}
\begin{tikzpicture}[scale=10,tdplot_main_coords]
\pgfmathsetmacro{\rvec}{.8}
\pgfmathsetmacro{\phivec}{40}
\pgfmathsetmacro{\thetavec}{50}
\pgfmathsetmacro{\omegavec}{60}

\coordinate (O) at (0,0,0);
\draw[thick,->] (0,0,0) -- (1,0,0) node[anchor=north east]{$x$};
\draw[thick,->] (0,0,0) -- (0,1,0) node[anchor=north west]{$y$};
\draw[thick,->] (0,0,0) -- (0,0,1) node[anchor=south]{$z$};
\tdplotsetcoord{P}{\rvec}{\phivec}{\thetavec}
\draw[-stealth,color=red] (O) -- (P) node[above left] {$(r,z)$};
\draw[dashed, color=red] (O) -- (Pxy);
\draw[dashed, color=red] (P) -- (Pxy);
\tdplotdrawarc[color=red,->]{(O)}{0.2}{0}{\thetavec}{anchor=north}{$\zeta$}

\coordinate (er) at ($(P)+0.4*({cos(\thetavec)},{sin(\thetavec)},0)$);
\coordinate (etheta) at ($(P)+0.4*({-cos(90-\thetavec)},{sin(90-\thetavec)},0)$);
\coordinate (ez) at ($(P)+0.4*(0,0,1)$);
\draw[-stealth] (P) -- (er) node[below right] {$\hat{e}_\mu$};
\draw[-stealth] (P) -- (etheta) node[below right] {$\hat{e}_\eta$};
\draw[-stealth] (P) -- (ez) node[right] {$\hat{e}_\xi$};

\coordinate (Omega) at ($(P)+0.2*(-.5,1.5,2)$);
\draw[-stealth,color=blue] (P) -- (Omega) node[above right] {$\vec{\Omega}$};
\draw[dashed,color=blue] (Omega) -- ($(Omega)+0.2*(0,0,-2)$) -- (P);
\tdplotdrawarc[color=blue,->]{(P)}{0.2}{\thetavec}{\thetavec+\omegavec}{anchor=north}{$\omega$}

\tdplotsetthetaplanecoords{\phivec}
\tdplotsetrotatedcoords{\thetavec}{270}{0}
\tdplotsetrotatedcoordsorigin{(P)}
\tdplotdrawarc[tdplot_rotated_coords,color=blue,->]{(0,0,0)}{0.2}{0}{\thetavec}{anchor=south west}{$\theta$}

\end{tikzpicture}
\caption{Cylindrical space-angle coordinate system showing the position $(r,z)$ and direction-of-travel $\vec{\Omega}$.}
\label{fig:CylindricalCoordinateSystem}
\end{figure}
%
Since the $\hat{e}_\mu$-axis is always in the $r$-direction, the coordinate system describing $\vec{\Omega}$ depends upon the spatial position.

In \RZ\ geometry, we assume there is no solution variation in the azimuthal direction, i.e.
\begin{flalign}
\frac{\partial \psi}{\partial \zeta} & \equiv 0,
\end{flalign}
%
which simplifies the streaming term to
\begin{flalign}
\vec{\Omega} \vd \grad \psi & = \frac{\mu}{r} \frac{\partial}{\partial r} (r \psi) + \xi \frac{\partial \psi}{\partial z} - \frac{1}{r} \frac{\partial}{\partial \omega} (\eta \psi),
\label{eq:RZStreamingTerm}
\end{flalign}
%
\noindent where the third term is the angular derivative term.

The common method of handling the angular discretization (developed by Morel and Montry~\cite{MorelAnalysisEliminationFluxDip} and described in detail in Section~\ref{sec:MethodsRZ}) is to discretize the direction-of-travel using level-symmetric angular quadrature, implement an approximation for the angular derivative, and sweep through the angular quadrature directions propagating angular flux information through the directions of travel. This method for handling the angular derivative has been implemented in other research that was focused on spatial discretizations. Following is a brief overview of that work.

Bailey et al.~\cite{BaileyDFEMCylindrical} derived the PWLD transport equation for \RZ\ geometry on arbitrary polygonal meshes and showed that their method is accurate in the diffusion limit. Bailey~\cite{BaileyDissertation} performed an asymptotic diffusion limit analysis for PWLD in \RZ\ geometry on an arbitrary polygonal mesh and found the leading order angular flux is isotropic. The PWLD scheme also displayed $O(h^2)$ convergence rates in several test problems, as expected.

Morel et al.~\cite{MorelLLDrz} performed an asymptotic diffusion limit analysis for the LLD method in \RZ\ geometry on a triangular mesh. The LLD equations satisfy a lumped linear continuous (LLC) diffusion discretization to leading order on the mesh interior. They also numerically demonstrate a $O(h^2)$ spatial convergence in the thick diffusion limit.

Efforts to maintain positivity on non-orthogonal meshes in \RZ\ geometry have been successful. Morel et al.~\cite{MorelLBLD} derived a lumped BLD scheme for quadrilaterals that ``is conservative, preserves the constant solution, preserves the thick diffusion limit, behaves well with unresolved boundary layers, and gives second-order accuracy in both the transport and thick diffusion-limit regimes.''

At this time, only Woods and Palmer~\cite{Woods2018RZJCTT} have applied the HO DFEM in \RZ\ geometry on meshes with curved surfaces. That work is presented in this dissertation in support of the present research objectives.

%%%%%%%%%%%%%%%%%%%%%%%%%%%%%%%%%%%%%%%
\subsection{Spherical Symmetry in \RZ\ Geometry}
\label{sec:SphericalSymmetryIntro}
Radiation-hydrodynamics is a multiphysics problem comprised of a thermal radiation problem coupled with a hydrodynamics problem. In cases with an axis of symmetry, a three-dimensional problem can be reduced to one or two spatial dimensions. For instance, a hohlraum could be modeled in \RZ\ geometry where a physical (or assumed) symmetry exists along the $z$-axis. This dimension reduction may save on computational memory. Some ICF applications employ a spherical target inside of a hohlraum. It is important to preserve the spherical symmetry of the target while using \RZ\ geometry. Since the hydrodynamics equations are being solved in \RZ\ geometry, it may be convenient to solve the radiation transport equations in \RZ\ geometry. When solving in \RZ\ geometry, the transport equation must also preserve spherical symmetry.

\begin{comment}
Despite the convenience of describing the analytical position within a sphere using spherical coordinates, the spatial discretization becomes much more complicated. It can be more convenient to solve a spherical problem using \RZ\ geometry because there are more angular derivatives in the streaming term of the radiation transport equation in spherical coordinates than in cylindrical. The increased complexity is analogous to the increased complexity of using \RZ\ geometry from Cartesian coordinates.
\end{comment}

There has been relatively little research in preserving spherical symmetry with the radiation transport equation in \RZ\ geometry. As previously mentioned, in hydrodynamics, Dobrev et al.~\cite{DobrevHOAxisymmetric} demonstrated this spherical symmetry preservation. In radiation diffusion, Brunner et al.~\cite{BrunnerSphericalsymmetry} demonstrated conditional spherical symmetry preservation. To date, we are only aware of one published article discussing the ability of the \RZ\ transport equation to preserve 1-D spherical symmetry. Chaland and Samba~\cite{Chaland2016SphericalSymmetry} qualitatively demonstrated spherical symmetry preservation in a void with an initial condition of scalar flux in the center of the ``sphere''. They demonstrated that ray-effects are created from the \SN\ transport equations in a void, whereas their alternative angular discretization qualitatively preserves the 1-D spherical symmetry. We follow the direction of Brunner et al.~\cite{BrunnerSphericalsymmetry} by determining the spherical symmetry preservation by using the method of manufactured solutions (MMS)~\cite{Lingus} with a manufactured solution in spherical geometry.

%%%%%%%%%%%%%%%%%%%%%%%%%%%%%%%%%%%%%%%
\subsection{Diffusion Limit}
\label{sec:DiffLimitIntro}
In high energy density physics (HEDP) problems, a particle can have a very small mean free path (mfp) compared to the size of a spatial zone. The mfp is the inverse of the total cross section $\Lambda=\sigma_t^{-1}$. Therefore, when the total macroscopic cross section, $\sigma_t$, becomes very large, a span of length can be described as being many mfp's long. These problems are called ``optically thick''. We require the transport equation (Equation~\ref{eq:RadTransport}) to resolve solutions in the optically thick regime. We can assess the behavior of Equation~\ref{eq:RadTransport} as the problem becomes increasingly optically thick by performing an asymptotic analysis. Specifically, a small factor $\varepsilon$ can be used to scale the physical processes of Equation~\ref{eq:RadTransport}:
\begin{flalign}
\vec{\Omega} \vd \grad \psi + \frac{\sigma_t}{\varepsilon} \psi & = \frac{1}{4 \pi} \left(\frac{\sigma_t}{\varepsilon} - \varepsilon \sigma_a \right) \int_{4 \pi} \psi d \Omega^\prime + \varepsilon S_0,
\end{flalign}
%
\noindent where arguments have been dropped for brevity. Then, as $\varepsilon \rightarrow 0$, the mean free path $\Lambda = \varepsilon / \sigma_t \rightarrow 0$. In this limit, the problem is said to be optically thick and diffusive. It can be shown~\cite{LarsenAsymptoticSoln1,MalvagiAsymptoticAnalysis} that this scaled analytic transport equation limits to the analytic radiation diffusion equation to $O(\varepsilon^2)$. The diffusion equation provides accurate solutions to optically thick and diffusive problems in the problem interior. Determining that the analytic transport equation limits to the analytic diffusion equation is physically meaningful. Optically thick and diffusive problems that may be typically solved with the radiation diffusion equation may also be solved using the radiation transport equation.

The diffusion equation has substantially fewer degrees of freedom for the same spatial discretization making it much quicker to solve numerically. This benefit is accompanied by some drawbacks. The diffusion equation is inaccurate in optically thin problems and near highly absorbing regions, voids, and strong material discontinuities~\cite{D&H}. However, the transport equation can provide accurate solutions in all of these regimes, including the diffusive regions. Therefore, this dissertation employs the transport equation to solve problems that are optically thick and diffusive.

Alhough the analytic transport equation limits to the analytic diffusion equation in the diffusion limit, it is important that the discretized transport equation limits to a meaningful discretized diffusion equation. The diffusion limit analysis has been applied to spatially discretized transport problems. The LD and lumped LD (LLD) methods were shown to have the diffusion limit in one-dimension by Larsen and Morel~\cite{LarsenAsymptotic} and Larsen~\cite{LarsenAsymptoticDiffusionLimit}, respectively, which is one of the reasons for their continued popularity in literature. In multiple dimensions, the LD method generally fails in the diffusion limit~\cite{BorgersAsymptoticDiffLimit}, but the LLD~\cite{MorelLLDrz, MorelLLDTetrahedral}, BLD~\cite{Adams_Disc_FEM_Thick_Diff}, and fully lumped BLD (FLBLD)~\cite{AdamsDFEMDiffLimit} methods possess the diffusion limit. Lumping the BLD equations in curvilinear geometry improves the character of the transport solution in the diffusion limit as well~\cite{PalmerCurvilinearTransport, MorelLBLD}. Attila has been demonstrated to solve problems in the diffusion limit using TLD, despite having the potential for negative fluxes~\cite{AttilaUsersManual}.

The PWLD has also been popular for its performance in the diffusion limit. Stone and Adams~\cite{StonePLFEM} concluded the PWLD on unstructured meshes should behave as well as BLD. Bailey et al.~\cite{BaileyDFEMCylindrical, BaileyDissertation} corroborated this and extended the PWLD method to RZ geometry and concluded this method has the diffusion limit. Bailey et al.~\cite{BaileyBLDFEM} then introduced piece-wise BLD (PWBLD) that allows for more curvature in a solution and has properties favorable to having the diffusion limit, although they do not perform the asymptotic analysis to be conclusive.

Palmer and Adams~\cite{PalmerCurvilinearTransport} and Palmer~\cite{PalmerDissertation} solved the \RZ\ \SN\ equations (discussed in Section~\ref{sec:RZGeometryIntro}) using BLD, mass lumped BLD (MLBLD), surface lumped BLD (SLBLD), FLBLD, and simple corner balance (SCB) methods but found that only the FLBLD and SCB methods are accurate in the thick diffusion limit, as predicted by their asymptotic diffusion limit analysis. They determined that the support points must be sufficiently ``local'' to achieve a reasonable discretization of the diffusion equation.

Guermond and Kanschat~\cite{GuermondDiffLimit} performed an analytical diffusion limit analysis for arbitrary-order spatially discretized transport equations that are exactly integrated in angle. Haut~\cite{Haut2018PersonalComm} followed this work to demonstrate that the arbitrary-order spatially discretized transport equation also has the diffusion limit using the discrete ordinates method to discretize in angle. Woods et al.~\cite{WoodsHoDgfemXyCurved} numerically demonstrate that two-dimensional Cartesian HO finite elements trend toward the diffusion limit; these results are presented in the absence of a rigorous analytic prediction of the diffusion limit behavior. Their study was performed without a source iteration acceleration technique so they were unable to perform calculations for highly optically thick media. Their work was performed on a quadrilateral mesh with curved surfaces.
%Their results indicated their method may converge toward the diffusion limit at $O(\varepsilon)$ rather than $O(\varepsilon^2)$, which may have occurred due to the use of curved surfaces~\cite{Adams_Disc_FEM_Thick_Diff} or unrefined boundary layers.


%%%%%%%%%%%%%%%%%%%%%%%%%%%%%%%%%%%%%%%
\subsection{Source Iteration Acceleration}
\label{sec:SourceIterationAccelerationIntro}
Radiation transport is integral to applications such as inertial confinement fusion and astrophysics. These problems often have materials that can be exceptionally optically thick (material regions are many mean-free-path lengths across) and diffusive (highly scattering). A transport solver should be capable of performing accurate and efficient calculations on such problems. The source iteration (SI) method~\cite{Lewis_Comp_Methods_Neu_Trans} is commonly employed to solve the transport equation. The algorithm
\begin{subequations}
\begin{flalign}
\vec{\Omega} \vd \grad \psi_m^{(\ell+1/2)} + \sigma_t \psi_m^{(\ell+1/2)} & = \frac{1}{4 \pi} \sigma_s \phi^{(\ell)} + S_0 \label{eq:DSASITransport}\\
\phi^{(\ell+1/2)} & = \sum_m w_m \psi^{(\ell+1/2)}_m \label{eq:DSASIPhiHalf} \\
\phi^{(\ell+1)} & = \phi^{(\ell+1/2)} \label{eq:DSASIUnacceleratedUpdate}
\end{flalign}
\label{eqs:SourceIteration}
\end{subequations}

\noindent describes the calculation of the angular flux using the previous scalar flux iterate followed by an update to the scalar flux using all of the angular fluxes. All spatial locations are coupled within each angle, and all angles are coupled at each spatial location. This coupling lends itself to a two-stage iterative method. One iteration of Equations~\ref{eqs:SourceIteration} is shown within the boxed region of Figure~\ref{fig:SolutionFlowDiagram}. Specifically, Equation~\ref{eq:DSASITransport} falls within ``Radiation transport solve'' where all spatial locations are coupled within each angle. Equations~\ref{eq:DSASIPhiHalf}~and~\ref{eq:DSASIUnacceleratedUpdate} are within ``Update scalar flux'', where all angles are coupled at each spatial position. This iterative process is continued until the scalar flux, $\phi$, converges to within some defined tolerance.

The SI algorithm can converge arbitrarily slowly in optically thick and diffusive problems~\cite{LarsenStableDSATheory} resulting in impractical computational times and the possibility of false convergence. One alternative method to avoid these slowly converging problems is to solve the problem using the radiation diffusion equation instead of the transport equation. The diffusion equation can resolve solutions in optically thick and diffusive media and does not require source iteration. However, the diffusion equation cannot resolve solutions in highly absorbing media, optically thin media, nor media with strong material discontinuities including vacuum boundaries~\cite{D&H}. We retain the ability to obtain accurate solutions in a wide variety of media, including those that the diffusion equation cannot resolve, by solving the transport equation despite the potential drawback of slowly converging SI.

In order to preserve the transport solution on optically thick and diffusive problems, we must reduce the SI computational time in one of two ways. We may refine the mesh until the optical thickness of a typical mesh cell is on the order of a mean-free-path to effectively solve an optically thin problem in each mesh zone. However, this option may not be computationally efficient because it introduces a large number of degrees of freedom to the problem, thereby increasing the solution time. Alternatively, acceleration techniques may be applied to the SI to compensate for slow convergence.

Several SI acceleration schemes have been developed. Diffusion synthetic acceleration (DSA) is a very common method. The Wareing-Larsen-Adams (WLA) method~\cite{WareingDSADFEM}, accelerates the source iterations but its effectiveness was found to degrade as cells become optically thick~\cite{WarsaFullyConsistentLDDSA}. Another alternative, modified four-step (M4S)~\cite{AdamsFastIterativeMethods}, is effective in 1-D but is only conditionally stable in 2-D for unstructured meshes. Using DSA, Larsen~\cite{LarsenStableDSATheory} and Larsen and McCoy~\cite{LarsenStableDSANumericalResults} showed that one-dimensional LD is unconditionally accelerated. Wareing et al.~\cite{WareingDSADFEM} and Adams and Martin~\cite{AdamsDSADFEM} extended unconditionally-accelerating DSA to two-dimensions. It is well known that the DSA algorithm can lose its effectiveness in multidimensional problems with strong material discontinuities. However, this degradation can be eliminated by preconditioning a Krylov iterative method with DSA~\cite{WarsaKrylovDSA}. In the present work, we focus on the DSA discretization method and its efficiency on homogeneous problems, and therefore avoid the need for a preconditioned Krylov solver.

The general DSA algorithm is
\begin{subequations}
\begin{flalign}
\vec{\Omega} \vd \grad \psi_m^{(\ell+1/2)} + \sigma_t \psi_m^{(\ell+1/2)} & = \frac{1}{4 \pi} \left(\sigma_s \phi^{(\ell)} + S_0 \right) \tag{\ref{eq:DSASITransport}} \\
\phi^{(\ell+1/2)} & = \sum_m w_m \psi_m^{(\ell+1/2)} \tag{\ref{eq:DSASIPhiHalf}} \\
-\grad \vd D \grad \varphi^{(\ell+1/2)} + \sigma_a \varphi^{(\ell+1/2)} & = \sigma_s \left(\phi^{(\ell+1/2)} - \phi^{(\ell)} \right) \label{eq:DSASIEquation} \\
\phi^{(\ell+1)} & = \phi^{(\ell+1/2)} + \varphi^{(\ell+1/2)} \label{eq:DSASIPhiEnd}
\end{flalign}
\label{eqs:DSAAlgorithm}
\end{subequations}

\noindent where $\phi^{(\ell+1/2)}$ is the radiation transport scalar flux solution at iteration $(\ell+1/2)$ prior to the DSA solve. Specifically, we solve for Equation~\ref{eq:DSASITransport} for each of the angular fluxes, $\psi^{(\ell+1/2)}_m$ at a half-step. Then, we perform a weighted summation of the angular fluxes to obtain the scalar flux at the half step, $\phi^{(\ell+1/2)}$. For comparison, the unaccelerated source iteration method can then be performed by Equation~\ref{eq:DSASIUnacceleratedUpdate}. The DSA algorithm then solves the DSA equation (Equation~\ref{eq:DSASIEquation}), a diffusion equation with a modified source. Finally, the summation of the solution to the DSA equation and the scalar flux at the half step becomes the scalar flux solution at the end of the iteration.

The discretization scheme of the Equation~\ref{eq:DSASIEquation} is very important and can determine whether a DSA method is stable or not. Fully consistent DSA schemes are unconditionally convergent and use the zeroeth and first moments of the discretized transport equation resulting in the mixed-form of the discretized diffusion equation~\cite{WarsaFullyConsistentLDDSA}. Recently, Wang and Ragusa~\cite{WangRagusaDSA} developed the modified interior penalty (MIP) form for the DSA equations. Considered a partially consistent scheme, their derivation only utilizes the zeroeth moment of the discretized transport equation and results in the second-order partial differential equation form of the diffusion equation. The MIP DSA method was developed from the interior penalty (IP) form of a discretized diffusion equation and was originally applied to HO DFEM \SN\ transport on triangular meshes. The IP method is not stable for optically thick media so it was combined with a spatial discretization they derived --- diffusion conforming form (DCF) --- that is stable in the optically thick regime. The MIP DSA equations switch between the IP and DCF methods and were demonstrated to be an effective SI acceleration scheme for HO DFEM methods. Wang and Ragusa's Fourier analysis demonstrated the effectiveness of this DSA scheme and corroborated it with numerical results. It has since been used with PWLD on arbitrary polygonal meshes~\cite{TurcksinDiscontinuousDSA} and later with BLD~\cite{TurcksinDSABLD}. Woods et al.~\cite{WoodsDSA} implemented the MIP DSA equations and investigated the performance on meshes with curved surfaces; demonstrating that the method remains unconditionally stable.

The MIP DSA equations were derived with homogeneous Dirichlet boundary conditions according to Kanschat~\cite{KanschatDGViscousIncompressFlow}. It has been acknowledged that homogeneous Dirichlet boundary conditions may degrade performance~\cite{WangDissertation}. Other DSA implementations utilize Robin (also referred to as Marshak, zero incident current, mixed, or vacuum) boundary conditions~\cite{WarsaFullyConsistentLDDSA,AdamsDSADFEM}. This allows for a nonzero DSA correction solution on the problem boundaries, thereby allowing a nonzero update to the transport scalar flux solution. In this dissertation, we present an extension of the work done by Woods et al.~\cite{WoodsDSA} that demonstrates a rapidly-convergent iterative solver for the transport equation spatially discretized with the HO DFEM, which employs the MIP DSA equations with homogeneous Robin boundary conditions.

%%%%%%%%%%%%%%%%%%%%%%%%%%%%%%%%%%%%%%%
\subsection{Research Objectives}
\label{sec:ResearchObjectivesIntro}
The research objectives of this dissertation are to:
\begin{itemize}
\item{derive, implement, and characterize the transport equation in \XY\ geometry using HO DFEM on meshes with curved surfaces,}
\item{derive, implement, and characterize the transport equation in \RZ\ geometry using HO DFEM on meshes with curved surfaces, and}
\item{derive, implement, and characterize the modified interior penalty (MIP) diffusion synthetic acceleration (DSA) method using Robin boundary conditions.}
\end{itemize}

%%%%%%%%%%%%%%%%%%%%%%%%%%%%%%%%%%%%%%%
\subsection{Outline}
\label{sec:OutlineIntro}
The remainder of this dissertation is organized as follows. In Section~\ref{sec:Methods}, we discuss the HO DFEM, meshes with curved surfaces, and the solution methods. In Section~\ref{sec:Results}, we show the results as a result of implementing these methods. Finally, we make some concluding remarks in Section~\ref{sec:Conclusions}.

%\bibliographystyle{apalike}
%\bibliography{Thesis_bib}


\end{document}