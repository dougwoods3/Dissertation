\documentclass{article}

\usepackage{OSUDissertation}

\begin{document}

\setlength{\abovedisplayskip}{5pt}
\setlength{\belowdisplayskip}{5pt}

%%%%%%%%%%%%%%%%%%%%%%%%%%%%%%%%%%%%%%%
\section{Supplementary Results in \XY\ Geometry}
In this section, we present additional results in \XY\ geometry to supplement the work done in Woods~\cite{WoodsThesis}. Some of this work appears in Woods et al.~\cite{Woods2018HOSnXY}.

%%%%%%%%%%%%%%%%%%%%%%%%%%%%%%%%%%%%%%%
\subsection{Uniform Infinite Medium}
We solved a uniform infinite medium problem on a high-order mesh that has sufficient curvature to have incident and outgoing angular fluxes on some mesh surfaces for the same angular direction $\vec{\Omega}$. The mesh is overlaid on top of the solution in Figure~\ref{fig:uim3PointQ4Q3R1p4S12}.

\begin{figure}[!htb]
\centering
\includegraphics[scale=0.4,trim={20pt 500pt 20pt 60pt},clip]{../../Research/graphics/uim3PointQ4Q3R1p4S12}
\caption{Uniform infinite medium scalar flux solution.}
\label{fig:uim3PointQ4Q3R1p4S12}
\end{figure}

%%%%%%%%%%%%%%%%%%%%%%%%%%%%%%%%%%%%%%%
\subsection{Spatial Convergence Study}
We use the manufactured solution
\begin{flalign}
\psi(x,y,\mu,\eta) & = (1-\mu^2)(1-\eta^2) \sin(\alpha \pi x) \cos(\beta \pi y)
\label{eq:GleicherMMS}
\end{flalign}

\noindent on a \emph{highly} curved mesh with $\alpha=4$ and $\beta=7/2$. The solution using $p=4$ and $S_{12}$ level-symmetric angular quadrature is shown in Figure~\ref{fig:Gleicherp4S123Pointv2Q3Q21R} with the coarsest mesh overlaid. Figure~\ref{fig:GleicherS123PointQ3Q2R1Z2p2g-1} shows the errors from a spatial convergence study with select reference lines. For coarser meshes, we observe degraded convergence rates as low as $O(1/2)$. As the mesh is refined, the convergence rates tend toward $O(p+1)$. This is likely the effect of zones tending to have fewer surfaces that have both incident and outgoing intensities for any given $\vec{\Omega}_m \vd \hat{n}(\vec{x})$ as the mesh is refined. Figure~\ref{fig:GleicherS123PointQ3Q2R1Z2p2g-1} reveals that the error for any given mesh refinement is reduced with higher-order finite elements. Inspection of the first mesh refinement step shows that the spatial convergence rate for higher-order finite elements is not as degraded as lower-order methods.

This particular solution has a scattering and external source so the solution is not discontinuous within the zone but rather has a function whose first derivative is discontinuous along some surfaces. The $O(1/2)$ convergence rate is similar to the convergence rate of solutions that are discontinuous within a mesh zone~\cite{WangDGFEMConvergence}. In this work, we numerically compute all integrals with Gaussian quadrature, which does not accurately integrate functions with discontinuous derivatives.

\begin{figure}[!htb]
\includegraphics[scale=0.4,trim={20pt 500pt 20pt 60pt},clip]{../../Research/graphics/Gleicherp4S123Pointv2Q3Q21R}
\caption{MMS on a \emph{highly} curved mesh.}
\label{fig:Gleicherp4S123Pointv2Q3Q21R}
\end{figure}

\begin{figure}[!htb]
\centering
\begin{tikzpicture}
  \begin{axis}[
    width=0.85\textwidth,
    %height=4.6cm,
    grid=major,
    xlabel={$\sqrt{\text{N}_\text{zones}}$},
    ylabel={L$^2$-error},
  	xmode=log,
  	ymode=log,
  	xmin=3,xmax=1e2,
  	ymin=5e-2,ymax=1.5e1,
  	%legend style={at={(1.2,1.1)},anchor=north east},
  	]
\addplot[mark=*, dashed, mark size=2, draw=black, mark options={solid, fill=black}] table [x=un, y=fe1]{../../Research/graphics/GleicherS123PointQ3Q2R1Z2p2g-1.dat};
\addlegendentry{$p=1$}
\addplot[mark=triangle*, dashed, mark size=3, draw=red, mark options={solid, fill=red}] table [x=un, y=fe2]{../../Research/graphics/GleicherS123PointQ3Q2R1Z2p2g-1.dat};
\addlegendentry{$p=2$}
\addplot[mark=square*, dashed, mark size=2, draw=blue, mark options={solid, fill=blue}] table [x=un, y=fe3]{../../Research/graphics/GleicherS123PointQ3Q2R1Z2p2g-1.dat};
\addlegendentry{$p=3$}
\addplot[mark=diamond*, dashed, mark size=3, draw=magenta, mark options={solid, fill=magenta}] table [x=un, y=fe4]{../../Research/graphics/GleicherS123PointQ3Q2R1Z2p2g-1.dat};
\addlegendentry{$p=4$}

\addplot[black] table[row sep=crcr] {
4 10 \\
8 7.07 \\};
\node[text=black, fill=white, anchor=south west] at (axis cs:6.25,8) {$O(1/2)$};
\addplot[black] table[row sep=crcr] {
4 3 \\
8 0.5303 \\};
\node[text=black, anchor=east] at (axis cs:6,0.8){$O(5/2)$};
\addplot[black] table[row sep=crcr] {
32 1 \\
64 0.25 \\};
\node[text=black, anchor=west] at (axis cs:48,0.6) {$O(2)$};
\addplot[black] table[row sep=crcr] {
16 1.3 \\
32 0.1625 \\};
\node[text=black, anchor=west] at (axis cs:25,0.45){$O(3)$};

  \end{axis}
  
\end{tikzpicture}
\caption{Errors between the manufactured and DFEM solutions. Reference lines depict order of spatial convergence $O(n) \equiv O\left(\left(N_\text{zones} \right)^{n/2} \right)$.}
\label{fig:GleicherS123PointQ3Q2R1Z2p2g-1}
\end{figure}

%%%%%%%%%%%%%%%%%%%%%%%%%%%%%%%%%%%%%%%
\subsection{Alternating Incident Boundary Condition}
Figure~\ref{fig:AlternatingIncident3PointQ4Q3R1} demonstrates the solution to an alternating incident angular flux problem similar to what appears in Woods~\cite{WoodsThesis} and Woods et al.~\cite{WoodsHoDgfemXyCurved}. Adams~\cite{AdamsDFEMDiffLimit} and Palmer and Adams~\cite{PalmerCurvilinearTransport} demonstrate that transport solutions in the optically thick diffusion limit that have anisotropic and/or discontinuous incident fluxes can permit unphysical oscillations in the interior solution. The material is defined by $\sigma_t=1000$, $\sigma_s=0.999 \sigma_t$, and $S_0=0$. We solved this using $4^\text{th}$-order finite elements and $S_{12}$ level-symmetric angular quadrature. These solutions demonstrate that a boundary layer is created by the discontinuous incident flux and propagates into the problem interior. These oscillations dip below zero and are non-phyiscal. The oscillations do not reach the region of interest so we alter this problem.

\begin{figure}[!htb]
\centering
\begin{subfigure}{\textwidth}
\centering
\begin{tikzpicture}
\node at (0,0) {\includegraphics[scale=0.4,trim={20pt 500pt 20pt 60pt},clip]{../../Research/graphics/AlternatingIncident3PointQ4Q3R1}};
\draw[red,line width=2pt] (5.71,2.15) -- (5.87,2.15);
\draw[red,line width=2pt] (6.03,2.15) -- (6.2,2.15);
\draw[red,line width=2pt] (6.38,2.15) -- (6.5,2.15);
\draw[red,line width=2pt] (6.6,2.06) -- (6.6,1.74);
\draw[red,line width=2pt] (6.6,1.42) -- (6.6,1.1);
\draw[red,line width=2pt] (6.6,0.78) -- (6.6,0.46);
\end{tikzpicture}
\caption{Scalar flux; red bars denote incident boundary locations.}
\label{subfig:AlternatingIncident3PointQ4Q3R1}
\end{subfigure}
\begin{subfigure}{\textwidth}
\centering
\begin{tikzpicture}
\node at (0,0) {\includegraphics[scale=0.4,trim={20pt 500pt 20pt 60pt},clip]{../../Research/graphics/AlternatingIncident3PointQ4Q3R1log}};
\draw[red,line width=2pt] (5.71,2.15) -- (5.87,2.15);
\draw[red,line width=2pt] (6.03,2.15) -- (6.2,2.15);
\draw[red,line width=2pt] (6.38,2.15) -- (6.5,2.15);
\draw[red,line width=2pt] (6.6,2.06) -- (6.6,1.74);
\draw[red,line width=2pt] (6.6,1.42) -- (6.6,1.1);
\draw[red,line width=2pt] (6.6,0.78) -- (6.6,0.46);
\end{tikzpicture}
\caption{Log of scalar flux solution; red bars denote incident boundary locations.}
\label{subfig:AlternatingIncident3PointQ4Q3R1log}
\end{subfigure}
\caption{Alternating incident flux problem solution.}
\label{fig:AlternatingIncident3PointQ4Q3R1}
\end{figure}

We solve a similar problem with the incident boundary condition locations in the upper right of the problem. Figure~\ref{fig:AlternatingIncident3Pointv2Q4R1mfp250} shows the scalar flux solution with red bars denoting the incident flux locations. The material is defined by $\sigma_t=250$, $\sigma_s=0.999 \sigma_t$, and $S_0=0$. We solved this using $4^\text{th}$-order finite elements and $S_{12}$ level-symmetric angular quadrature. The total cross-section was reduced for this problem so that there are some mesh zones that the zones in the bottom right region have comparable mfp sizes compared to the problem described by Figure~\ref{fig:StrongScatterProblem}. We observe oscillations that dip below zero throughout the solution, however they are preferentially in the zones that have larger mfp sizes (i.e., left side of the mesh). Many of the zones comprising the spiral are large in this sense and exhibit the oscillations. An interesting result is that these oscillations are propagated away from the boundaries of spiral region.

\begin{figure}[!htb]
\centering
\begin{subfigure}{\textwidth}
\centering
\begin{tikzpicture}
\node at (0,0) {\includegraphics[scale=0.4,trim={20pt 500pt 20pt 70pt},clip]{../../Research/graphics/AlternatingIncident3Pointv2Q4R1mfp250}};
\draw[red,line width=2pt] (-2.5,2.2) -- (-0.95,2.22);
\draw[red,line width=2pt] (0.6,2.2) -- (2,2.2);
\draw[red,line width=2pt] (3,2.2) -- (3.2,2.2);
\draw[red,line width=2pt] (-2.55,2.15) -- (-2.55,1.9);
\draw[red,line width=2pt] (-2.55,1.66) -- (-2.55,1.4);
\draw[red,line width=2pt] (-2.55,1.14) -- (-2.55,0.84);
\end{tikzpicture}
\caption{Scalar flux; red bars denote incident boundary locations.}
\end{subfigure}
\begin{subfigure}{\textwidth}
\centering
\begin{tikzpicture}
\node at (0,0) {\includegraphics[scale=0.4,trim={20pt 500pt 20pt 70pt},clip]{../../Research/graphics/AlternatingIncident3Pointv2Q4R1mfp250log}};
\draw[red,line width=2pt] (-2.5,2.2) -- (-0.95,2.22);
\draw[red,line width=2pt] (0.6,2.2) -- (2,2.2);
\draw[red,line width=2pt] (3,2.2) -- (3.2,2.2);
\draw[red,line width=2pt] (-2.55,2.15) -- (-2.55,1.9);
\draw[red,line width=2pt] (-2.55,1.66) -- (-2.55,1.4);
\draw[red,line width=2pt] (-2.55,1.14) -- (-2.55,0.84);
\end{tikzpicture}
\caption{Log of the scalar flux; red bars denote incident boundary locations.}
\end{subfigure}
\caption{Scalar flux solution to alternating incident flux problem.}
\label{fig:AlternatingIncident3Pointv2Q4R1mfp250}
\end{figure}





\FloatBarrier

















\end{document}