\documentclass{article}

\usepackage{OSUDissertation}

\begin{document}

%%%%%%%%%%%%%%%%%%%%%%%%%%%%%%%%%%%%%%%
\section{Supplementary Results in \XY\ Geometry}
In this section, we present additional results in \XY\ geometry to supplement the work done in Woods~\cite{WoodsThesis}. Some of this work appears in Woods et al.~\cite{Woods2018HOSnXY}.

We solved a uniform infinite medium problem on a high-order mesh that has sufficient curvature to have incident and outgoing angular fluxes on some mesh surfaces for the same angular direction $\vec{\Omega}$. The mesh is overlaid on top of the solution in Figure~\ref{fig:uim3PointQ4Q3R1p4S12}.

\begin{figure}[!htb]
\centering
\includegraphics[scale=0.4,trim={20pt 500pt 20pt 60pt},clip]{../../Research/graphics/uim3PointQ4Q3R1p4S12}
\caption{Uniform infinite medium scalar flux solution.}
\label{fig:uim3PointQ4Q3R1p4S12}
\end{figure}

Figure~\ref{fig:AlternatingIncident3PointQ4Q3R1} demonstrates the solution to an alternating incident angular flux problem similar to what appears in Woods~\cite{WoodsThesis} and Woods et al.~\cite{WoodsHoDgfemXyCurved}. Adams~\cite{AdamsDFEMDiffLimit} and Palmer and Adams~\cite{PalmerCurvilinearTransport} demonstrate that transport solutions in the optically thick diffusion limit that have anisotropic and/or discontinuous incident fluxes can permit unphysical oscillations in the interior solution.

\begin{figure}[!htb]
\centering
\begin{subfigure}{\textwidth}
\centering
\includegraphics[scale=0.4,trim={20pt 500pt 20pt 60pt},clip]{../../Research/graphics/AlternatingIncident3PointQ4Q3R1}
\caption{Scalar flux solution.}
\label{subfig:AlternatingIncident3PointQ4Q3R1}
\end{subfigure}
\begin{subfigure}{\textwidth}
\centering
\includegraphics[scale=0.4,trim={20pt 500pt 20pt 60pt},clip]{../../Research/graphics/AlternatingIncident3PointQ4Q3R1log}
\caption{Log of scalar flux solution.}
\label{subfig:AlternatingIncident3PointQ4Q3R1log}
\end{subfigure}
\caption{Alternating incident flux problem solution.}
\label{fig:AlternatingIncident3PointQ4Q3R1}
\end{figure}





\FloatBarrier

















\end{document}